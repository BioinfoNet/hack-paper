\section*{Introduction}\label{introduction}
\addcontentsline{toc}{section}{Introduction}

Our Core message is to present a clear outlook on the adoption of Open
Science practice by African-based scientists, starting with Kenya.

Add an introduction of the Open Science. To cite an article, use
\cite{Schlegel2016}. All the bibliographies should be added to
\texttt{library.bib} in the BibTeX format. See the example in
\texttt{library.bib}.

Since this will more or less be like a review article, we will need to
identify the various subsections based on the topics we need to cover in
the review. See
\href{http://journals.plos.org/ploscompbiol/article?id=10.1371/journal.pcbi.1005619}{this
article} for some tips \cite{Mensh2017}.

\textbf{Placeholder Text for the Background} WILL UPDATE TO INTRODUCTION

Open, collaborative and reproducible research -- Open Science -- has a
great potential for advancing science. However, the training in our
local universities does not equip students with the tools to practice
open science. However, to work in the open and collaborate, your
collaborators should be equipped to use the tools that you use. The main
barrier to working open, therefore, is the lack of awareness of the
collaboration tools and the skills required to utilize these tools.
Therefore, to fill the gap through an open science community, funded by
a Mozilla Mini-grant -- OpenScienceKE, we are promoting open science
among bioinformatics students and researchers in the Nairobi area by
training using this model: sensitize, train, hack and collaborate. This
model first sensitizes on open science practices through seminars,
trains on open science tools through workshops, facilitates hands-on
application of the tools through hackathons, and finally fosters a
community of open science enthusiasts through meetups.

OpenScienceKE sought to address the following problems: the lack of
awareness of open science practices and tools within the Bioinformatics
community in Kenya; the poor adoption of open science practices in
Bioinformatics; and the absence of research to establish the state of
affairs in adopting open science in Kenya. From the OpenScienceKE
hackathon, we managed to create an open resource that the students could
use to figure out where they can cost-effectively publish open access.
In addition, through literature search and data mining, we observed a
growing interest in open science practices in Kenya but the lack of
awareness and skills hinder the adoption. The use of preprints for
research dissemination haven't caught up in Kenya; out of the 20,069
papers downloaded from bioRXiv, only 18 have Kenyan authors, a majority
of which are as a result of international (16) collaborations. We also
observed a lack of incentives and policy in academic and research
institutions to support open science. The fear of being scooped and the
competitive spirit within the scientific community are also major
barriers to working in the open.

The first iteration of the model which focused on academic institutions
set the foundation for next phase: promote the open and reproducible
science in research institutions. This model provides the framework for
the adoption of open science practices within the institution and others
in the future. As genomic research data generated in Africa grows, there
is a need for the adoption of open science practices in data storage,
reproducible pipelines and collaborative research. We propose this
approach, which develops the necessary infrastructure within research
institutions, and builds human capacity through the model: sensitize,
train, hack and collaborate. Promotion of open science in Africa
recognizes the future direction of research and OpenScienceKE is growing
the culture and practice in the research acti

\subsection*{Status of Open Science in Kenya: Literature
search}\label{status-of-open-science-in-kenya-literature-search}
\addcontentsline{toc}{subsection}{Status of Open Science in Kenya:
Literature search}

The report from the above team will be useful for writing the
introduction, as well as providing materials to be used in the
discussion. In fact, we can use their resource to weave the whole paper
together.

In this section, and the introduction, we will conduct a review of the
status of open science in the country. - What kind of resources are
available to support open science - Are there policies or incentives for
open science practices? - What kind of training activities have been
conducted to promote and train students and researchers on open science
tools and practices? - etc

\subsubsection{Proposed subsections and
content}\label{proposed-subsections-and-content}

The subsections will be based on the findings about the status of open
science in Kenya.

\begin{enumerate}
\def\labelenumi{\arabic{enumi}.}
\tightlist
\item
  Resources to support open science are insufficient. The resources
  include:
\end{enumerate}

\begin{itemize}
\tightlist
\item
  ICT infrastructure. KENET provides infrastructure for open science
  training.
\item
  Institutional repositories. Universities and Research institutes have
  repositories to store publications. Also repositories to store
  research data. Do all institutions have these repositories?
\end{itemize}

\begin{enumerate}
\def\labelenumi{\arabic{enumi}.}
\setcounter{enumi}{1}
\tightlist
\item
  Lacking policies and incentives from the government, local funders and
  institutions. These policies and incentives lack in open science
  practices such as:
\end{enumerate}

\begin{itemize}
\tightlist
\item
  Research data management. Apart from
  \href{https://cgspace.cgiar.org/handle/10568/63496}{ILRI's data
  management policy}, do other institutions have these policies?
\item
  Open access publishing
\end{itemize}

\begin{enumerate}
\def\labelenumi{\arabic{enumi}.}
\setcounter{enumi}{2}
\tightlist
\item
  Limited training on Open Science. Awareness on Open Science has
  previously been through:
\end{enumerate}

\begin{itemize}
\tightlist
\item
  Open access week in universities. (Do research institutions
  participate in Open Access week?)
\item
  Workshops such as OpenScienceKE Workshop in 2018
\item
  Others?
\end{itemize}

\section*{Data Mining Section}\label{data-mining-section}
\addcontentsline{toc}{section}{Data Mining Section}

This will be a data analysis section. The title of this section will
depend on the results of your analysis. For example, if Kenyan
researchers are not publishing open access, we will need to understand
why that is the case. The solution may lie in the cost of publishing,
and that is how the resource created by Open Access options team is
useful.

We address questions like: - What is the publishing trend by Kenyan
researchers - Are they publishing open access, and how has this changed
over the years? - Are Kenyan researchers embracing pre-prints (BioRXiv,
AriRXiv, ResearchGate, F1000Research). Who is driving the adoption of
pre-prints? Local researchers or foreign collaborators? - What are the
collaboration trends? Are Kenyan researcher collaborating locally or
internationally?

\subsubsection{Proposed title and
content}\label{proposed-title-and-content}

The title here should indicate whether \emph{Kenyan researchers are
embracing open science practices} as per the findings of the Data Mining
team. This section's discussion may therefore focus on supporting this
claim based on:

\begin{enumerate}
\def\labelenumi{\arabic{enumi}.}
\tightlist
\item
  The publishing trend of Kenyan researchers. The trend indicates that
  open access publishing is popular. The popular journals where Kenyan
  researcher publish are open access.
\item
  Pre-prints. Pre-prints are yet to be widely adopted by Kenyan
  scientists and foreign collaborators drive the adoption of pre-prints
  among Kenyan authors
\item
  Open data and Code.
  \textbf{\emph{\href{https://github.com/BioinfoNet/Data-mining/issues/11}{This
  is yet to be completed}. We need to find out whether Kenyan authors
  are making the data and code available and accessible, and whether
  they are adhering to the FAIR principles}}
\item
  Collaboration trends. Are Kenyan authors collaborating from within or
  without? (based on pre-prints from bioRXiv, we see that Kenyan authors
  are mostly collaborating internationally. Is this the case from the
  rest of literature obtained during data mining?) 5 What else?
\end{enumerate}

\section*{But Publishing Open Access is
Expensive}\label{but-publishing-open-access-is-expensive}
\addcontentsline{toc}{section}{But Publishing Open Access is Expensive}

The article processing charge is the main barrier to publishing open
access, in addition to the obsession with impact factors. However, for
early career scientists and students, especially in developing
countries, most publishers offer waivers and subsidies but few are
aware. In this section, we explore some of the avenues to publishing
open access at low cost.

To address this problem, we created a resource that can guide ECR and
students on where they publish open access, and at low cost. We also
provide information on how they can still be open when they publish in
paywalled journals, eg via the green route.

\subsubsection{Proposed content of
Section}\label{proposed-content-of-section}

\begin{itemize}
\tightlist
\item
  Low cost open access publishing is available.
\end{itemize}

Here our discussion will indicate that there are publishers offering
waivers and subsidies for developing countries (low income and lower
middle income countries). Waiver is either Full or 50\% of the APCs.
Sometimes, waivers are offered on a case-by-case basis. The resource
created will create awareness on where to publish at low cost in an open
access way.

\section*{Figures}\label{figures}
\addcontentsline{toc}{section}{Figures}

You can add the figures as follows:

\begin{figure}[htbp]
\centering
%% \includegraphics{./figure01.pdf}
\caption{Figure 1}
\end{figure}

And you can have it referenced as a figure

\begin{quote}
\textbf{Box 1} To highlight of defining some key concepts in Open
science without disrupting the flow of the articles, you can use a quote
format.
\end{quote}

\section*{Discussion}\label{discussion}
\addcontentsline{toc}{section}{Discussion}

What do the results mean? How does your results fit to the current
literature? How do they compare to other similar studies?

\section*{Conclusions}\label{conclusions}
\addcontentsline{toc}{section}{Conclusions}

What is the take-home message from this article? What are the
recommendations? - The need for a framework to guide the adoption of
open science practices. Take note of the barriers and provide
recommendations. - The need for low-cost publishing - The need for
policies on open science that can be implemented by various
institutions. For example, we can provide a template that can be adopted
by most institutions

\section{Acknowledgement}\label{acknowledgement}

We acknowledge the support from \ldots{} and the contribution from
Mozilla Science Lab and KENET, and ICIPE.

Use this section to acknowledge funding and resource contributions to
the project.
