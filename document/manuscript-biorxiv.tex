\documentclass[10pt,letterpaper]{article}
\usepackage[top=0.85in,left=2.75in,footskip=0.75in,marginparwidth=2in]{geometry}

% use Unicode characters - try changing the option if you run into troubles with special characters (e.g. umlauts)
\usepackage[utf8]{inputenc}

% clean citations
\usepackage{cite}

% hyperref makes references clicky. use \url{www.example.com} or \href{www.example.com}{description} to add a clicky url
\usepackage{nameref,hyperref}

% line numbers
\usepackage[right]{lineno}

% improves typesetting in LaTeX
\usepackage{microtype}
\DisableLigatures[f]{encoding = *, family = * }

% text layout - change as needed
\raggedright
\setlength{\parindent}{0.5cm}
\textwidth 5.25in 
\textheight 8.75in

% Remove % for double line spacing
%\usepackage{setspace} 
%\doublespacing

% use adjustwidth environment to exceed text width (see examples in text)
\usepackage{changepage}

% adjust caption style
\usepackage[aboveskip=1pt,labelfont=bf,labelsep=period,singlelinecheck=off]{caption}

% remove brackets from references
\makeatletter
\renewcommand{\@biblabel}[1]{\quad#1.}
\makeatother

% headrule, footrule and page numbers
\usepackage{lastpage,fancyhdr,graphicx}
\usepackage{epstopdf}
\pagestyle{myheadings}
\pagestyle{fancy}
\fancyhf{}
\rfoot{\thepage/\pageref{LastPage}}
\renewcommand{\footrule}{\hrule height 2pt \vspace{2mm}}
\fancyheadoffset[L]{2.25in}
\fancyfootoffset[L]{2.25in}

% use \textcolor{color}{text} for colored text (e.g. highlight to-do areas)
\usepackage{color}

% define custom colors (this one is for figure captions)
\definecolor{Gray}{gray}{.25}

% this is required to include graphics
\usepackage{graphicx}

% use if you want to put caption to the side of the figure - see example in text
\usepackage{sidecap}

% use for have text wrap around figures
\usepackage{wrapfig}
\usepackage[pscoord]{eso-pic}
\usepackage[fulladjust]{marginnote}
\reversemarginpar

% document begins here
\begin{document}
\vspace*{0.35in}

% title goes here:
\begin{flushleft}
{\Large
\textbf\newline{Towards Open and Reproducible Genomic Research: Lessons from OpenScienceKE}
}
\newline
% authors go here:
\\
Caleb Kibet \textsuperscript{1,3,*},
Author 2\textsuperscript{2},
Author 3\textsuperscript{1},
Author 4\textsuperscript{1},
Author 5\textsuperscript{2},
Author 6\textsuperscript{2},
OpenScienceKE\textsuperscript{3}
\\
\bigskip
\bf{1} International Center of Insect Physiology and Ecology
\\
\bf{2} Affiliation B
\\
\bf{3} Members of the OpenScienceKE community.
\\
\bigskip
* ckibet@icipe.org

\end{flushleft}

\section*{Abstract}

Open, collaborative and reproducible research – Open Science – has a great potential for advancing science. However, the training in our local universities does not equip students with the tools to practice open science. However, to work in the open and collaborate, your collaborators should be equipped to use the tools that you use. The main barrier to working open, therefore, is the lack of awareness of the collaboration tools and the skills required to utilize these tools. Therefore, to fill the gap through an open science community, funded by a Mozilla Mini-grant – OpenScienceKE, we are promoting open science among bioinformatics students and researchers in the Nairobi area by training using this model: sensitize, train, hack and collaborate. This model first sensitizes on open science practices through seminars, trains on open science tools through workshops, facilitates hands-on application of the tools through hackathons, and finally fosters a community of open science enthusiasts through meetups. 

OpenScienceKE sought to address the following problems: the lack of awareness of open science practices and tools within the Bioinformatics community in Kenya; the poor adoption of open science practices in Bioinformatics; and the absence of research to establish the state of affairs in adopting open science in Kenya. From the OpenScienceKE hackathon, we managed to create an open resource that the students could use to figure out where they can cost-effectively publish open access. In addition, through literature search and data mining, we observed a growing interest in open science practices in Kenya but the lack of awareness and skills hinder the adoption. The use of preprints for research dissemination haven’t caught up in Kenya; out of the 20,069 papers downloaded from bioRXiv, only 18 have Kenyan authors, a majority of which are as a result of international (16) collaborations. We also observed a lack of incentives and policy in academic and research institutions to support open science.  The fear of being scooped and the competitive spirit within the scientific community are also major barriers to working in the open. 

The first iteration of the model which focused on academic institutions set the foundation for next phase: promote the open and reproducible science in research institutions. This model provides the framework for the adoption of open science practices within the institution and others in the future. As genomic research data generated in Africa grows, there is a need for the adoption of open science practices in data storage, reproducible pipelines and collaborative research. We propose this approach, which develops the necessary infrastructure within research institutions, and builds human capacity through the model: sensitize, train, hack and collaborate. Promotion of open science in Africa recognizes the future direction of research and OpenScienceKE is growing the culture and practice in the research active region. 

% now start line numbers
\linenumbers

\input manuscript

% the * after section prevents numbering

\section*{Acknowledgments}
We thank KENET for providing us with an ample environment for our hackathon. 

\nolinenumbers

%This is where your bibliography is generated. Make sure that your .bib file is actually called library.bib
\bibliography{library}

%This defines the bibliographies style. Search online for a list of available styles.
\bibliographystyle{abbrv}

\end{document}

